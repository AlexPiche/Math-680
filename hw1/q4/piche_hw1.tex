\documentclass{article}
\title{MATH 680 Homework 1 Fall 2015}
\date{\today}
 
\begin{document}
\maketitle

This homework is due on Wednesday, September 16 at 11:59pm. Provide both pdf, R
files (and \LaTeX file for exercise 3). The naming convention is 
yourlastname\_hw1.pdf, yourlastname\_hw1.R and yourlastname\_hw1.lyx

\cite{friedman2001greedy}

1. (20\%) Walk through the following steps:
• Sign up on https://github.com/.
• Create a repository (repo).
• Add me as a collaborator (github ID: emeryyi).
• DownloadandinstallSourceTreehttps://www.sourcetreeapp.com/oralternativelyGitHub Desktop https://desktop.github.com/ as the graphical user interface (GUI) client.
• Clone the repository your created on github.com to your local folder using the GUI client.
• Add an R file in the local folder, then stage and commit changes.
• Push the changes to your origin repo on github.com
• Make a new branch and switch to the new branch using “checkout”.
• Make some changes in the new branch, then stage and commit.
• Switch to the master (old) branch using “checkout”.
• Make some changes in the old branch (but make sure they do not conflict with the changes in the new branch), then stage and commit.
• Merge the old branch with the new branch, and then commit.
• Push all the changes to the remote server.
• Provide the hyperlink and a screen shot of your github repo in a pdf file, which reflects all the commits, branching and merging you made in your repo.
2. (80\%) In this exercise, we reproduce the “random function generator” (RFG) model by [1] (the paper is included, see page 19 section 6.1) in R. The RFG model generates very complicated data with non-linearity and higher-order interactions. The data generated by RFG can be used for
1
testing the performance of many fully non-parametric regression-based learning methods such as kernel support vector machine (KSVM), gradient boosting, random forests and others.
The idea is to generate a data frame with N = 100 observations of simulation data {yi,xi}N1 according to
yi =f(xi)+εi,
where εis are independent generated from the normal distribution ε ∼ N(0, 1). In the data frame y takes up the first column, and the p = 10 dimensional vector x takes up the rest columns. Hence the data frame will be 100 × 11. Each row represents one observation. The f functions is randomly generated as a linear combination of functions {g }20:
20
f(x) = 􏰧 algl(zl), (1) l=1
where coefficients {a }20 are randomly generated from a uniform distribution a ∼ U[−1, 1]. Each l1l
gl(zl) is a function of a randomly selected pl-size subset (sub-vector) of the p-dimensional variable x, with p = 10, and the size of each subset pl is randomly chosen by pl = min(⌊1.5 + r⌋ , p), and r is generated from an exponential distribution r ∼ Exp(0.5) with mean 2. Each zl is a sub-vector of x defined as
zl = {xW (j)}pl , (2) l j=1
where each Wl is an independent permutation of the integers {1, . . . , p}. Each function gl(zl) is an pl-dimensional Gaussian function:
􏰋1T􏰌
gl(zl) = exp −2(zl − μl) Vl(zl − μl) , (3)
where each of the mean vectors {μ }20 is randomly generated from the same distribution as that l1
of the input variables x. The pl × pl covariance matrix Vl is also randomly generated by
Vl = UlDlUTl , (4)
where Ul is a pl × pl uniformly distributed random orthonormal matrix and Dl = diag{d1l...dpll}. The variables djl are randomly generated from a uniform distribution 􏰉djl ∼ U[0.1, 2.0]. We gen- erated x from joint normal distribution x ∼ N(0, Ip ) with p = 10.
Hints:
• In this exercise, you will probably use the following R functions:
l1
– \Sexpr{rexp}, runif, rnorm – floor
– qr.Q, qr
– diag
– sample
2
• To generate the uniformly distributed random orthonormal matrix Ul, please see
this post
http://math.stackexchange.com/questions/138512/sampling-q-uniformly-where-qtq-i.
Specifically, you need to generate a Gaussian matrix G, and use QR decomposition
to find the corresponding Q matrix, which will be a uniformly distributed random
orthonormal matrix. Use R functions qr.Q and qr to achieve this. Or much easier,
you can just use the genQ function in the package bootSVD
https://cran.r-project.org/web/packages/bootSVD/ index.html to do the same
job.
• If it is your first time using R, learn the above functions using their
online manuals before starting the actual coding.
3. (Extra credit 40\%) Walk
through the following steps:
• Download and install LYX. To make LYX work
correctly, Mac users may also need to install MacTEX https://tug.org/mactex/
while the Windows users may need to install MikTEX http://miktex.org/.
•
Reproduce this document using LYX, make sure that all text contents,
mathematical formu- las, references and typesetting (bullet, bold text, url,
title, etc.) are accurately reproduced. However, the layout of the document do
not need to be exactly the same. (paper margin, line space, etc.).

\bibliographystyle{unsrtnat} 
\bibliography{Untitled} 
\end{document}